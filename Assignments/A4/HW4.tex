\documentclass[10pt,a4paper]{article}
\usepackage[a4paper,left=2.5cm,right=2.5cm,bottom=3cm,top=2.5cm]{geometry}
\usepackage[latin1]{inputenc}
\usepackage{amsmath,dsfont,bbm,setspace}
\usepackage[longnamesfirst,nonamebreak]{natbib}
\usepackage{longtable,booktabs,pdfpages}
\usepackage[english]{babel}
\usepackage{eurosym,multirow,hyperref}
\usepackage[refpage]{nomencl}
\usepackage[lastexercise]{exercise}

\hypersetup{
    colorlinks,%
    citecolor=blue,%
    filecolor=black,%
    linkcolor=blue,%
    urlcolor=black}

\newcommand{\Pe}{\mathbb{P}}
\newcommand{\E}{\mathbb{E}}
\newcommand{\Lik}{\mathcal{L}}
\newcommand{\D}{\mathcal{D}}
\newcommand{\N}{\mathcal{N}}
\newcommand{\C}{\mathcal{C}}
\newcommand{\R}{\mathcal{R}}
\newcommand{\1}{\mathds{1}}
\newcommand{\ow}{\overline{w}}

%\renewcommand{\labelitemi}{}

\begin{document}
\doublespacing

\title{\textcolor{blue}{Linear Panel Data}}
\maketitle

The goal of this exercise is to apply linear panel data techniques. You are allowed to use the pre-programmed OLS estimator in your preferred statistical package.


\begin{Exercise}[title=Data]
Load the data ``Koop - Tobias'', which comes from
Koop and Tobias (2004) Labor Market Experience Data.   
(See Koop, G. and J. Tobias, "Learning About Heterogeneity in Returns to Schooling," Journal of Applied Econometrics, 19, 2004, pp. 827-849.

The data file is in two parts. The first file contains the panel of 17,919 observations on the Person ID and 4 time-varying variables. The second file contains time invariant variables for the individual or the 2,178 households. 

Variables in the file may be changing over time 
\begin{itemize}
\item PERSONID = Person id (ranging from 1 to 2,178),
\item EDUC = Education,
\item LOGWAGE = Log of hourly wage,
\item POTEXPER = Potential experience,
\item TIMETRND = Time trend.
\end{itemize}

or time invariant
\begin{itemize}
\item ABILITY = Ability,
\item MOTHERED = Mother's education,
\item FATHERED = Father's education,
\item BRKNHOME = Dummy variable for residence in a broken home,
\item SIBLINGS = Number of siblings
\end{itemize}
Represent the panel dimension of wages for 5 randomly selected individuals. 
\end{Exercise}


\begin{Exercise}[title=Random Effects]
We are interested in
\begin{equation}
 LOGWAGE_{it} = \alpha_i + \beta_{1} EDUC_{it} + \beta_{2} POTEXPR_{it} + \epsilon_{it} 
\end{equation}
Where $\alpha_i$ is a random effect. Estimate the random effect model under the normality assumption of the disturbance terms. 
\end{Exercise}


\begin{Exercise}[title=Fixed Effects Model]
We are interested in
\begin{equation}
 LOGWAGE_{it} = \alpha_i + \beta_1 EDUC_{it} + \beta_2 POTEXPR_{it} + \epsilon_{it} 
\end{equation}
Where $\alpha_i$ is individual fixed effect. Estimate the following estimators
\begin{itemize}
 \item Between Estimator
 \item Within Estimator
 \item First time difference Estimator
\end{itemize}
Compare the estimates of $\beta_1$ and $\beta_2$ under the different models.
\end{Exercise}

\begin{Exercise}[title=Understanding Fixed Effects]
In the rest of the assignment, we consider only a random selected 100 individuals. We are interested in
\begin{equation}
 LOGWAGE_{it} = \alpha_i + \beta_1 EDUC_{it} + \beta_2 POTEXPR_{it} + \epsilon_{it} 
\end{equation}
Where $\alpha_i$ is individual fixed effect.
\begin{itemize}
 \item Write and optimize the likelihood associated to the problem and estimate the individual fixed effect parameters
 \item Run a regression of estimated individual fixed effets on the invariant variables.
 \item The standard errors in the previous may not be correctly estimated. Explain why, and propose an alternative method to compute standard errors.
\end{itemize}
\end{Exercise}



\end{document}




