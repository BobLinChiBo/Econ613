\documentclass[10pt,a4paper]{article}
\usepackage[a4paper,left=2.5cm,right=2.5cm,bottom=3cm,top=2.5cm]{geometry}
\usepackage[latin1]{inputenc}
\usepackage{amsmath,dsfont,bbm,setspace}
\usepackage[longnamesfirst,nonamebreak]{natbib}
\usepackage{longtable,booktabs,pdfpages}
\usepackage[english]{babel}
\usepackage{eurosym,multirow,hyperref}
\usepackage[refpage]{nomencl}
\usepackage[lastexercise]{exercise}

\hypersetup{
    colorlinks,%
    citecolor=blue,%
    filecolor=black,%
    linkcolor=blue,%
    urlcolor=black}

\newcommand{\Pe}{\mathbb{P}}
\newcommand{\E}{\mathbb{E}}
\newcommand{\Lik}{\mathcal{L}}
\newcommand{\D}{\mathcal{D}}
\newcommand{\N}{\mathcal{N}}
\newcommand{\C}{\mathcal{C}}
\newcommand{\R}{\mathcal{R}}
\newcommand{\1}{\mathds{1}}
\newcommand{\ow}{\overline{w}}

%\renewcommand{\labelitemi}{}

\begin{document}
\doublespacing

\title{\textcolor{blue}{Data}}
\maketitle

The goal of this exercise is to familiarize with large and realistic data, and learn basic techniques for data manipulation and descriptive statistics. We will use three datasets:
\begin{itemize}
\item datstu: is an administrative data on students from junior high school applying for admission to senior high school through a centralized application system. Students apply to specific academic programs within a school and can submit a ranked list of up to six programs. 
\begin{itemize}
\item score: student test score
\item agey: student age
\item male: student male
\item schoolcode1: first school 
\item schoolcode2: second school
\item choicepgm1: first program 
\item schoolpgm2: second program
\item jssdistrict: 
\end{itemize}
\item datjss: the longitude ($point_x$) and latitude ($point_y$) of each district (jssdistrict).
\item datsss: school name, school code, district, longitude and latitude.
\end{itemize}

\begin{Exercise}[title=Missing data]
Report the following statistics
\begin{itemize}
\item Number of students
\item Number of schools
\item Number of programs
\item Number of choices (school,program
\item Missing test score
\item Apply to the same school (different programs)
\item Apply to less than 6 choices
\end{itemize}
\end{Exercise}


\begin{Exercise}[title=Data]
Create a school level dataset, where each row corresponds to a (school,program) with the following variables:
\begin{itemize}
\item the district where the school is located
\item the latitude of the district
\item the longitude of the district
\item cutoff (the lowest score to be admitted)
\item quality (the average score of the students admitted)
\item size (number of students admitted)
\end{itemize}
\end{Exercise}


\begin{Exercise}[title=Distance]
\begin{itemize}
\item Using the formula
\begin{equation*}
dist(sss,jss) = \sqrt(69.172*(ssslong-jsslong)*cos(jsslat/57.3))^2+(69.172*(ssslat-jsslat))^2)
\end{equation*}
where ssslong and ssslat are the coordinates of the district of the school (students apply to), while jsslong and jsslat are the coordinates of the junior high school, calculate the distance between junior high school, and senior high school.
\end{itemize}
\end{Exercise}

\begin{Exercise}[title=Descriptive Characteristics]
Report the average and sd of the following variables for each ranked choice 
\begin{itemize}
\item Cutoff
\item Quality
\item Distance
\end{itemize}
Redo the same table, differentiating by student test score quantiles.
\end{Exercise}


\begin{Exercise}[title=Diversification]
Group schools by decile of selectivity (cutoffs), and compute for each individual the number of groups in the application. Redo this, by student test score (quantile)
\end{Exercise}


\end{document}




